\documentclass[conference]{IEEEtran}
\usepackage{cite}
\usepackage{amsmath,amssymb,amsfonts}
\usepackage{graphicx}
\usepackage{booktabs}
\usepackage{xcolor}
\usepackage[hyphens]{url}
\usepackage[
	colorlinks=true,       
	linkcolor=blue,        
	citecolor=blue,        
	urlcolor=blue          
]{hyperref}
\newcommand{\nextbreak}{\vfill\null\pagebreak}


\IEEEoverridecommandlockouts

\begin{document}
	
	\title{A Comprehensive Review of Digital Signal Processing and Machine Learning Techniques for Automated Egg Classification}
	
	\author{
		\resizebox{\textwidth}{!}{%
			\begin{tabular}{ccc}
				Christian James E. Apuya & Ray Simon L. Bantaculo & Robert Roy P. Salvo \\
				\textit{Department of Computer Engineering} & \textit{Department of Computer Engineering} & \textit{Department of Computer Engineering} \\
				University of Science and Technology & University of Science and Technology & University of Science and Technology \\
				of Southern Philippines & of Southern Philippines & of Southern Philippines \\
				Cagayan de Oro City, Philippines & Cagayan de Oro City, Philippines & Cagayan de Oro City, Philippines \\
				christianjamesapuya@gmail.com & bantaculoraysimon@gmail.com & salvo.robertroy@gmail.com \\
		\end{tabular}}
	}
	
	\maketitle
	
	\begin{abstract}
		Egg classification is a critical component in ensuring food safety, consumer trust, and efficient commercial distribution. Traditional grading approaches, such as manual candling, are labor-intensive, subjective, and often inconsistent. With recent advances in digital signal processing (DSP) and machine learning (ML), the automation of egg quality inspection has become increasingly viable. This review surveys eight peer-reviewed studies published between 2020 and 2025, highlighting novel methodologies ranging from image processing and acoustic signal analysis to hyperspectral and multispectral imaging. Each study is analyzed in terms of its methodological framework, contributions, limitations, and application context. The paper further synthesizes common challenges across the literature, including environmental variability, dataset scarcity, cost of equipment, and scalability. Proposed solutions, such as lightweight architectures, multimodal fusion, and transfer learning, are also discussed. This review aims to provide a comprehensive perspective to identify key gaps on the current state of egg classification research and identify promising directions for practical implementation.
	\end{abstract}
	
	\section{Introduction}
	Eggs represent one of the most consumed protein sources worldwide, making their quality assessment vital to food safety and market efficiency. The classification process typically involves detecting cracks, dirt, deformities, and assessing freshness. Traditional methods rely heavily on manual candling, where inspectors examine eggs under light sources. However, this approach is subjective, inconsistent, and impractical for high-throughput industrial applications 
	
	Over the last five years, researchers have turned to digital signal processing (DSP) and machine learning (ML) for developing automated, objective, and scalable solutions. Image processing, acoustic analysis, hyperspectral imaging, and deep learning models such as Convolutional Neural Networks (CNNs) and Vision Transformers (ViTs) have shown promising results in improving classification accuracy and efficiency. Despite these advances, challenges remain, particularly regarding environmental robustness, cost feasibility, and real-world scalability.  
	
	While several studies have focused on detecting cracks \cite{zhang2020cnn}, assessing internal quality \cite{li2021hsi}, or identifying surface defects \cite{chen2022lightweight}, there has been growing attention on egg size classification, which is essential for automated sorting systems in poultry industries. Singh et al. \cite{singh2022vision}, demonstrated a computer vision system that used morphological processing and edge detection to determine egg size and shape from digital images, achieving over 90\% accuracy in controlled environments. However, traditional vision-based methods like this often struggle with overlapping eggs, poor lighting, and inconsistent egg orientation—factors that significantly affect contour extraction and size estimation accuracy.
	
	To overcome these limitations, recent works have proposed lightweight CNN models optimized for embedded systems and multimodal imaging approaches combining visible and near-infrared data \cite{chen2022lightweight} \cite{garcia2024fusion}. Although these studies primarily address defect detection and freshness, their techniques—particularly image preprocessing, edge enhancement, and feature extraction—are highly relevant to improving the reliability of size-based classification. Meanwhile, advanced architectures such as Vision Transformers have shown potential for object feature recognition and could be adapted for more robust egg size categorization \cite{huang2025vit}.
	
	This paper provides a systematic review of eight representative studies from 2020 to 2025. We analyze their methodologies, highlight implementation challenges, discuss proposed solutions, and synthesize findings into a cohesive evaluation of the field.
	
	\section{Methodology of Reviewed Studies}
	
	\subsection{Image-Based Eggshell Defect Detection Using CNN (2020)}
	\cite{zhang2020cnn} developed a convolutional neural network (CNN)-based system for detecting cracks and surface dirt on eggshells. Their methodology combined DSP preprocessing with deep learning classification. Images of eggs were captured under controlled lighting, followed by Gaussian filtering to reduce background noise. Histogram equalization was applied to improve contrast, enhancing the visibility of fine cracks.  
	
	The CNN architecture, trained on thousands of labeled egg images, achieved over 95\% accuracy in distinguishing cracked from intact eggs. Compared to earlier Support Vector Machine (SVM)-based classifiers, the CNN demonstrated stronger generalization and robustness to minor variations in egg orientation.  
	
	However, the system required strict lighting control and high-quality images. In real-world farms with variable illumination, accuracy dropped significantly, highlighting a limitation of CNN models trained in laboratory conditions. Despite this, the study paved the way for computer vision-based egg classification systems.
	
	\subsection{Hyperspectral Imaging for Internal Quality Assessment (2021)}
	\cite{li2021hsi} explored hyperspectral imaging (HSI) for assessing internal egg freshness. The researchers captured spectral signatures of eggs across 400–1000 nm wavelengths. DSP techniques such as Savitzky–Golay smoothing and spectral normalization were used to reduce noise and baseline drift. Principal Component Analysis (PCA) reduced high-dimensional spectral data before classification with Partial Least Squares-Discriminant Analysis (PLS-DA).  
	
	Their system achieved 93\% accuracy in distinguishing fresh eggs from spoiled ones. Unlike visual inspection, HSI provided non-destructive insights into internal composition, particularly albumen degradation and yolk mobility.  
	
	Nevertheless, the study faced major barriers to adoption. Hyperspectral cameras are expensive and require precise calibration, making the system inaccessible for small farms. The computational overhead of processing hyperspectral data also raised scalability issues. Li et al. suggested integrating dimensionality reduction and cost-effective sensors as future improvements.
	
	\subsection{Acoustic Response Analysis for Crack Detection (2021)}
	\cite{wang2021acoustic} proposed an acoustic approach to detect cracked eggs. Using a mechanical tapping device, they recorded the resonance response of eggs with a microphone. Fast Fourier Transform (FFT) was applied to convert time-domain signals into frequency-domain features. Random Forest classifiers were then used to categorize eggs as intact or cracked.  
	
	The method achieved 92\% accuracy and offered advantages such as low cost and fast execution. Unlike image-based systems, it was unaffected by visual dirt or surface irregularities.  
	
	However, environmental noise presented a significant challenge. Acoustic systems required quiet, controlled conditions for reliable results. In noisy production environments, classification performance decreased, raising concerns about industrial applicability.
	
	\subsection{Computer Vision with Morphological Processing (2022)}
	\cite{singh2022vision} employed classical computer vision techniques for egg size and shape classification. Images were captured in conveyor-based systems, where eggs often overlapped. Preprocessing involved edge detection and morphological operations to isolate individual egg contours. Geometric features such as aspect ratio, roundness, and major/minor axes were extracted and fed into a Support Vector Machine (SVM).  
	
	The system classified eggs into standard market sizes with 90\% accuracy. Its main advantage was computational efficiency, requiring less power than deep learning methods.  
	
	Yet, the system struggled with overlapping eggs and poor segmentation in noisy environments. As conveyors introduced speed variability, feature extraction sometimes failed. The study highlighted the trade-off between computational efficiency and robustness in computer vision pipelines.
	
	\subsection{Lightweight Deep Learning for Embedded Systems (2022)}
	\cite{chen2022lightweight} developed a lightweight CNN model optimized for embedded systems such as Raspberry Pi. The architecture used depthwise separable convolutions to minimize computational cost. Preprocessing included contrast normalization and Gaussian noise reduction.  
	
	The model classified dirty versus clean eggs with 94\% accuracy while running at 12 frames per second on low-power devices. Unlike earlier deep learning systems, this design demonstrated feasibility for deployment in resource-constrained settings.  
	
	However, accuracy decreased slightly in low-light environments. The authors recommended combining lightweight CNNs with data augmentation techniques to improve real-world performance. Their work was significant in addressing cost and accessibility for small-scale farms.
	
	\subsection{X-ray Imaging for Internal Crack Detection (2023)}
	\cite{kumar2023xray} presented an X-ray imaging-based method to detect micro-cracks invisible to human eyes. Median filtering removed noise from raw images, and Gray-Level Co-occurrence Matrix (GLCM) was used to extract texture features. A Gradient Boosted Trees classifier then identified cracked versus intact eggs.  
	
	The system achieved 96\% accuracy, outperforming traditional visual inspection. Importantly, it detected cracks that would otherwise pass candling.  
	
	Despite its accuracy, the system faced limitations in cost, regulatory concerns, and safety issues associated with X-ray exposure. These barriers restricted deployment in small farms and retail contexts, suggesting its primary use for high-end industrial quality control.
	
	\subsection{Multispectral Fusion for Freshness Classification (2024)}
	\cite{garcia2024fusion} introduced a multimodal approach combining RGB imaging with near-infrared (NIR) spectroscopy. DSP preprocessing aligned multimodal inputs and corrected intensity variations. A deep neural network fused features from both modalities.  
	
	This multimodal design improved robustness, achieving 95\% accuracy under varying lighting conditions, outperforming unimodal systems.  
	
	However, the system required complex calibration and synchronization of multimodal sensors, increasing both cost and operational complexity. Despite this, the study demonstrated that integrating multimodal signals can significantly improve classification performance.
	
	\subsection{Vision Transformers for Egg Classification (2025)}
	\cite{huang2025vit} applied Vision Transformers (ViTs) to egg classification. Data augmentation and normalization were key preprocessing steps to prevent overfitting. The ViT achieved 97\% accuracy in detecting cracks, dirt, and freshness, surpassing CNN baselines.  
	
	The primary limitation was computational demand. ViTs require significant processing power, making them unsuitable for embedded applications without model compression. Nonetheless, this study highlighted the potential of transformer architectures in agricultural product classification.
	
	\subsection{A Computer Vision-Based Automatic System for Egg Grading and Defect Detection}
	In this study an advanced computer vision-based system was developed to automatically grade and detect defects in eggs by integrating deep learning and machine learning techniques. The researchers used 800 Hy-Line W-36 hens in a cage-free production system, from which eggs were collected daily and manually categorized into standard and non-standard groups based on size and surface quality. An image acquisition system was constructed using a Canon EOS 4000D camera, tripod, digital scale, and computer to capture and record both egg images and corresponding weights. The collected data underwent preprocessing, including noise removal, normalization, and clustering, to enhance image quality and ensure accurate feature extraction. For classification, the study employed a Real-Time Multitask Detection (RTMDet) model—a modern convolutional neural network (CNN) derived from YOLO architecture—enhanced with large-kernel depth-wise convolutions and soft label assignments to improve small-object detection and classification accuracy. The RTMDet model extracted key egg features such as major and minor axes for size and shape analysis, which were then used by a Random Forest regression algorithm to predict egg weight. The dataset, comprising 2100 egg images, was split into training and testing sets in a 4:1 ratio, and model performance was evaluated using precision, recall, F1-score, mean average precision (mAP), and coefficient of determination (R²). The best-performing variant, RTMDet-x, achieved a classification accuracy of 94.8\% and an R² value of 0.96 for weight prediction, demonstrating the system’s capability to perform simultaneous egg grading and weighting efficiently \cite{yang2023eggGrading}.
	
	\begin{table*}[htbp]
		\centering
		\caption{Summary of Reviewed Studies on Automated Egg Classification (2020–2025)}
		\label{tab:reviewed_studies}
		\begin{tabular}{|p{3cm}|p{3cm}|p{3cm}|p{3cm}|p{3cm}|}
			\hline
			\textbf{Study \& Year} & \textbf{Focus / Objective} & \textbf{Methods Used} & \textbf{Findings / Results} \\
			\hline
			\textbf{A. Image-Based Eggshell Defect Detection Using CNN (2020)} & Detect cracks and dirt on eggshells using CNN & DSP preprocessing (Gaussian filtering, histogram equalization); CNN classification on labeled images & 95\% accuracy; better generalization than SVM-based methods \\
			\hline
			\textbf{B. Hyperspectral Imaging for Internal Quality Assessment (2021)} & Assess internal freshness using hyperspectral imaging & HSI (400–1000 nm); Savitzky–Golay smoothing; PCA + PLS-DA classification & 93\% accuracy; non-destructive internal quality analysis \\
			\hline
			\textbf{C. Acoustic Response Analysis for Crack Detection (2021)} & Detect cracks through acoustic signals & Mechanical tapping; FFT for frequency-domain analysis; Random Forest classification & 92\% accuracy; low cost; unaffected by surface dirt \\
			\hline
			\textbf{D. Computer Vision with Morphological Processing (2022)} & Classify eggs by size and shape & Edge detection; morphological operations; geometric feature extraction; SVM classification & 90\% accuracy; computationally efficient; suitable for real-time use \\
			\hline
			\textbf{E. Lightweight Deep Learning for Embedded Systems (2022)} & Develop low-power CNN for small devices & Depthwise separable convolutions; preprocessing (contrast normalization, Gaussian noise reduction) & 94\% accuracy; runs at 12 FPS on Raspberry Pi; cost-effective \\
			\hline
			\textbf{F. X-ray Imaging for Internal Crack Detection (2023)} & Detect micro-cracks not visible to human eye & Median filtering; texture extraction using GLCM; Gradient Boosted Trees classification & 96\% accuracy; detects micro-cracks missed by candling \\
			\hline
			\textbf{G. Multispectral Fusion for Freshness Classification (2024)} & Combine RGB and NIR data for freshness assessment & DSP alignment and normalization; deep neural network for multimodal feature fusion & 95\% accuracy; robust to lighting variation; improved generalization \\
			\hline
			\textbf{H. Vision Transformers for Egg Classification (2025)} & Apply Vision Transformers (ViT) for defect detection & Data augmentation; normalization; ViT architecture for feature analysis & 97\% accuracy; outperforms CNN baselines \\
			\hline
			\textbf{I. Computer Vision-Based Automatic System for Egg Grading and Defect Detection (2025)} & Integrate deep learning and ML for grading and defect detection & Canon EOS camera; preprocessing (noise removal, normalization); RTMDet model; Random Forest regression for weight prediction & 94.8\% classification accuracy; $R^2 = 0.96$ for weight prediction; effective real-time performance \\
			\hline
		\end{tabular}
	\end{table*}
	
	\nextbreak
	
	\section{Challenges and Issues}
	
	Automated egg classification systems face a variety of significant challenges that affects their weidespread implementation in real-world settings. One of the most prominent issues is lighting variability in image-based methods. Systems such as those proposed by \cite{zhang2020cnn} and \cite{singh2022vision} rely heavily on consistent illumination to accurately detect cracks, dirt, or deformities on eggshells. Even minor variations in lighting conditions can cause misclassifications, as small surface features may be obscured or exaggerated. In practical farm environments, where natural light changes throughout the day and artificial lighting may be uneven or insufficient, which maintains optimal lighting conditions, which is often impractical and can lead to inconsistent performance and decreased reliability.
	
	Acoustic-based methods also face challenges, primarily due to environmental noise. The approach explored by \cite{wang2021acoustic} uses mechanical tapping and microphone-based detection to identify cracks. This can be sensitive to background sounds. Production lines are also often noisy due to machinery, and human activity, which can interfere with the detection of frequency-domain signals and result in false positives or misclassifications. This limits the scalability of acoustic systems in industrial settings, as additional noise suppression or soundproofing measures may be required, increasing complexity and cost.
	
	Another challenge is the high cost and operation complexity of advanced imaging technologies. Hyperspectral imaging, as described by \cite{li2021hsi}, and X-ray imaging, as shown by \cite{kumar2023xray}, provide highly accurate assessments of internal egg quality and micro-cracks. However, these systems require expensive hardware, precise calibration, and often specialized training for operators. The financial and logisital barriers associated with these technologies make them inaccessible for small-to-medium-sized farms.
	
	Limited datasets represent another major challenge across nearly all approaches. Many models are trained on relatively small, and lab-generated datasets that do not capture the full variability of real-world conditions such as egg size, color, surface texture, or environmental factors. Therefore, models may overfit to controlled experimental conditions, which results in poor generalization to new data or unseen farm environments. This problem is more present in deep learning approaches, which typically require large and diverse datasets.
	
	Finally, scalability to industrial operations remains a persistent challenge. Even models that demonstrate high accuracy in controlled laboratory conditions often struggle when integrated into high-throughput production lines. Factors such as overlapping eggs on conveyor belts, varying conveyor speeds, inconsistent orientations, and uncontrolled backgrounds can significantly degrade performance. These operational challenges highlight the gap between laboratory research and practical, real-world deployment, emphasizing the need for models that are both robust and adaptable to dynamic industrial environments. \\
	

	
	\section{Proposed Solutions}
	
	To address the challenges facing automated egg classification systems, researchers have proposed a variety of strategies aimed at improving robustness, scalability, and accessibility. One of the most widely adopted approaches is robust preprocessing and data augmentation. Techniques such as histogram equalization, spectral normalization, and geometric augmentation enhance system resilience against environmental conditions. These preprocessing steps are crucial in improving the reliability and consistency of vision-based classification systems.
	
	Lightweight architectures for embedded systems represent another promising solution. Models such as lightweight CNNs \cite{chen2022lightweight}, combined with mode compression techniques, enable real=time deployment on low-power devices like Raspberry Pi. By reducing computation requirements without significantly sacrificing accuracy, these architectures make automated egg inspection more accessible for small-to-medium-scale farmers.
	
	Multimodal fusion approaches have also shown significant potential in enhancing system robustness. By combining multiple sensor modalities such as standard RGB imaging with near-infrared (NIR) spectroscopy \cite{garcia2024fusion}. These systems can integrate complementary information from different sources. Multimodal designs reduce reliance on a single type of data. This helps soften the impact of variable lighting or environmental conditions and improving overall classification performance.
	
	Cost reduction through low-cost sensors is another key strategy. Researchers have explored replacing expensive hyperspectral or X-ray equipment with more affordable alternatives, which optimizes digitral signal processing pipelines to work effectively with inexpensive cameras and microphones. These efforts make advanced inspection technologies more feasible for widespread use, especially in small farms.

	Finally, hybrid DSP-ML pipelines offer a balanced solution that combines traditional signal processing techniques such as Fast Fourier Transform and morpholohical processing, with machine learning algorithms. These hybrid approaches improve interpretability, reduce computational load, and maintain high classification accuracy. By integrating classical DSP methods with modern machine learning, these pipelines provide an effective compromise between efficiency and accuracy, and addresses the many challenges in automated egg classification.
	
	\section{Conclusion}
	Between 2020 and 2025, research in egg classification has progressed remarkably through the integration of digital signal processing (DSP) and machine learning (ML). Techniques such as convolutional neural networks (CNNs), acoustic analysis, hyperspectral imaging, and Vision Transformers have enabled increasingly accurate and efficient systems for automated egg inspection. Each approach presents trade-offs, CNNs offer strong visual recognition, while acoustic and spectral methods provide insights into internal quality, though often at higher costs or computational demands. Despite these advances, challenges such as dataset scarcity, sensor affordability, and environmental variability still hinder large-scale deployment. Nonetheless, the growing focus on lightweight architectures, multimodal data fusion, and low-cost sensing technologies highlights a transition toward more practical and accessible solutions. Moreover, the adoption of federated learning and hybrid DSP-ML pipelines promises better generalization, interpretability, and adaptability across diverse agricultural contexts. Overall, the field is moving toward intelligent, sustainable, and scalable classification systems that could transform quality control in modern food production.
	
	\begin{thebibliography}{00}
		\bibitem{zhang2020cnn} Y. Zhang, et al., ``Eggshell crack detection using convolutional neural networks,'' \textit{Computers and Electronics in Agriculture}, vol. 170, 2020.
		\bibitem{li2021hsi} X. Li, et al., ``Hyperspectral imaging for egg freshness and internal quality assessment,'' \textit{Postharvest Biology and Technology}, vol. 179, 2021.
		\bibitem{wang2021acoustic} L. Wang, et al., ``Acoustic response analysis for egg crack detection,'' \textit{Biosystems Engineering}, vol. 201, 2021.
		\bibitem{singh2022vision} R. Singh, et al., ``Computer vision and morphological processing for egg size classification,'' \textit{Journal of Food Engineering}, vol. 310, 2022.
		\bibitem{chen2022lightweight} H. Chen, et al., ``Lightweight CNN for embedded egg defect detection,'' \textit{IEEE Access}, vol. 10, 2022.
		\bibitem{kumar2023xray} P. Kumar, et al., ``X-ray imaging for internal egg crack detection,'' \textit{Food Control}, vol. 145, 2023.
		\bibitem{garcia2024fusion} L. Garcia, et al., ``Fusion of RGB and NIR data for egg freshness classification,'' \textit{Sensors}, vol. 24, no. 6, 2024.
		\bibitem{huang2025vit} Z. Huang, et al., ``Vision Transformers for egg quality classification,'' \textit{Frontiers in Plant Science}, vol. 16, 2025.
		\bibitem{yang2023eggGrading} X. Yang, R. B. Bist, S. Subedi, and L. Chai, “A Computer Vision-Based Automatic System for Egg Grading and Defect Detection,” Animals, vol. 13, no. 14, p. 2354, Jan. 2023, \textit{doi: https://doi.org/10.3390/ani13142354.}
	\end{thebibliography}
	
\end{document}

